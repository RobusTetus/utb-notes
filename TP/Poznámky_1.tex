\documentclass{article}
\usepackage[a4paper]{geometry}
\usepackage{amsmath}
\title{Teorie Přenosu Informací}
\author{Robin Tetour}
\date{Květen 2023}

\begin{document}
\maketitle

\section{Teorie přenosu informací}
\subsection{Kvantita informace}
$I=n\times\log s$
kde:
\begin{itemize}
    \item $s$ počet symbolů se stejnou pravděpodobností výskytu
    \item $n$ celkový počet symbolů ve zprávě
\end{itemize}

\textbf{Typy událostí}
\begin{enumerate}
    \item Jistá událost
    \item Nemožná událost
    \item Opačná událost $\overline{A}$
    \item Událost A je součástí B $A\subset B$
    \item Události jsou rovnocenné $A=B$
    \item Průnik událostí $A\cdot B$
    \item Sjednocení $A+B$
    \item Rozdíl $A-B$ nebo $A\setminus B$
    \item Události jsou neslučitelné $A\cap B$
    \item Elementární událost
    \item Úplná soustava neslučitelných událostí $I=A_{1}+A_{2}+\ldots A_{n}$
\end{enumerate}

\subsection{Demorganovy zákony}
\begin{itemize}
    \item $\overline{\left(A+B\right)}=\overline{A}\cdot\overline{B}$
    \item $\overline{\left(A\cdot B\right)}=\overline{A}+\overline{B}$
\end{itemize}

\subsection{Geometrická pravděpodobnost}
$P\left(A\right)=\frac{\left|\Delta\right|}{\left|\Omega\right|}$
\paragraph{Uzavřená oblast $\Omega$ a v ní další $\Delta$, můžeme určit pravděpodobnost jevu $A$ (bod leží v obou oblastech)}

\subsection{Statistická definice pravděpodobnosti}
\begin{enumerate}
    \item $P\left(A\right)=\lim_{n\to\infty} \frac{f}{n}$
    \item $P\left(A\right)\approx\frac{f}{n}$ pro velká $n$
\end{enumerate}

\begin{itemize}
    \item $n$ počet pokusů
    \item $f$ počet výskytů
\end{itemize}

\subsection{Kombinatorika}
\begin{itemize}
    \item $V\left(k,n\right)=n\left(n-1\right)\ldots\left(n-k+1\right)$
    \item $P\left(n\right) =V\left(n,n\right) =n\left(n-1\right)\ldots1$
    \item $C(k,n)=1k!V(k,n)=1k!n(n-1)\ldots(n-k+1)=n!k!(n-k)!$
\end{itemize}

\begin{table}[h]
    \centering
    \begin{tabular}{c|c|c|c}
         & Variace & Permutace & Kombinace \\
        Bez opakování & $\frac{n!}{(n-k)!}$ & $n!$ &
        $\begin{pmatrix}
            n \\
            k \\
        \end{pmatrix}$\\
        S opakováním & $n^k$ & $\frac{(k_1+\ldots+k_n)!}{k_1!\times\ldots\times k_n!}$ & $\begin{pmatrix}
            n-k+1 \\
            k \\
        \end{pmatrix}$ \\
    \end{tabular}
    \caption{Vzorce kombinatoriky}
    \label{tab:komb}
\end{table}
\subsection{Entropie}
\begin{itemize}
    \item $H\left(x\right)=-P\left(x\right)\log_2P\left(x\right)$
\end{itemize}

\textit{Redundance jazyka}

$R=\frac{H}{H_{max}}$

\textbf{Informační hodnota}
\begin{itemize}
    \item $I\left(x\right)=\log_2\frac{1}{P(x)}=-\log_2P\left(x\right)$
    \item $I=\sum_{k=1}^{n}p_k\times \log_2 \frac{1}{p_k}$
    \item $I\left(E\right)=\log_2N$ = Hartleyho formule
    \begin{itemize}
        \item Hf definuje množství informace, které je nutné k charakterizaci jednoho prvku množiny E o N prvcích
    \end{itemize}
\end{itemize}
kde:
\begin{itemize}
    \item $N$ Počet možností/prvků
    \item $x$ Zpráva
\end{itemize}
\textbf{Sjednocení}
\begin{itemize}
    \item Informace $I(x)=I(A)+I(B)$
    \item Pravděpodobností $P(x)=P(A)\times P(B)$
\end{itemize}
\subsection{Kódování}
Rovnoměrný kód $\times$ Nerovnoměrný kód
\begin{enumerate}
    \item Shannon-fannovo
    \item Huffmanovo
\end{enumerate}

\begin{itemize}
    \item $\eta=\frac{H}{\overline{d}}\times 100\%$ Efektivita kódu
    \item $\overline{d}=\sum_{i=1}^{N}d_i\times p_i$ Průměrná délka kódového slova
\end{itemize}

\subsection{Bezpečnostní kódy}
\textbf{Paritní zabezečení}
Přidání jednoho bitu (0/1) na konec zprávy podle:
\begin{itemize}
    \item Sudá = sudý počet \textbf{jedniček}
    \item Lichá = lichý počet \textbf{jedniček}
\end{itemize}
\textbf{Hammingova Kostka (vzdálenost), Maticové zabezpečení}

\section{Test:}
\begin{itemize}
    \item Stromy + McMillan + Efektivita
    \item Převody
    \item Pravděpodobnost
    \item Entropie, Kombinatorika
    \item Zabezpečovací kódy
\end{itemize}

\subsection{Příklad}
\begin{enumerate}
    \item A1 - 0,2
    \item A2 - 0,26
    \item A3 - 0,12
    \item A4 - 0,16
    \item A5 - 0,12
    \item A6 - 0,06
    \item A7 - 0,05
    \item A8 - ?
\end{enumerate}

\subsection{Příklad}
Navrhněte kód pro zdrojovou abecedu, která obsahuje 7 symbolů a pravděpodobnosti výskytu jsou dány takto:
\begin{enumerate}
    \item 0,4
    \item 0,2
    \item 0,05
    \item 0,1
    \item 0,15
    \item 0,04
    \item ?
\end{enumerate}
\end{document}
